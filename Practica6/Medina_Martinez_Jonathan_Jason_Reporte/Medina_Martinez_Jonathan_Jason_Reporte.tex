\documentclass{article}
\usepackage[utf8]{inputenc}
\usepackage[spanish]{babel}
\usepackage{graphicx}
\usepackage{geometry}
\usepackage{enumerate}
\usepackage{titlesec}
\usepackage{float}
\usepackage{listings}
\usepackage{xcolor}
\usepackage{amsmath}

\geometry{letterpaper, margin = 1.5cm}


\newcommand{\codefontsize}{\fontsize{12}{12}}
\lstset{
	language=C,
	basicstyle=\codefontsize\ttfamily\color{white},
	keywordstyle=\color{blue},
	commentstyle=\color{green},
	stringstyle=\color{red},
	numbers=left,
	numberstyle=\tiny\color{black},
	stepnumber=1,
	numbersep=10pt,
	backgroundcolor=\color{blue!20!black},
	showspaces=false,
	showstringspaces=false,
	showtabs=false,
	tabsize=4,
	captionpos=b,
	breaklines=true,
	breakatwhitespace=true,
	escapeinside={(*@}{@*)},
	frame=single
}

%Datos de la Portada
\title{Introducción a la Programación \ Practica 3}
\author{Medina Martinez Jonathan Jason \ 2023640061}
\date{25 de Abril del 2023}

\begin{document} %Inicio del Documento
	
	\fontsize{12}{16}\selectfont
	
	\begin{figure}[t] %Logos Portada
		
		\includegraphics[width=2.5 cm]{Logo1.jpeg}
		\hfill
		\includegraphics[width=3 cm]{Logo2.png}
		
	\end{figure}
	
	\maketitle %Titulo Portada
	\newpage
	
	\tableofcontents %Indice
	\newpage
	
	\section{Objetivo}
	
	Desarrollar programas utilizando cadenas para el desplegado de informacion.
	
	\section{Introducción}
	
	La programación es una herramienta fundamental para el desarrollo de soluciones y el procesamiento de información. En esta práctica nos enfocaremos en el uso de cadenas para el despliegue de información, una habilidad importante en la programación de aplicaciones. A través de seis ejercicios, los cuales incluyen el cálculo del tamaño de una cadena, la identificación de letras repetidas, la separación de una cadena en sus componentes básicos, y la clasificación de caracteres, se busca que el estudiante fortalezca su conocimiento y habilidad en el uso de cadenas.
	
	\newpage
	\section{Desarrollo}
	Desarrolle los siguientes programas:
	
	\subsection{Problema 1}
	Programa que pida al usuario una cadena de texto y muestre el tamaño de la cadena.
	No deberá utilizar bibliotecas de funciones adicionales a stdio.h.
	
	\subsubsection{Programa1.c}
	
	\begin{lstlisting}

/**
* @file Programa1.c
* @author Medina Martinez Jonathan Jason (jmedinam1702@alumno.ipn.mx)
* @brief 
* @version 0.1
* @date 2023-05-01
* 
* @copyrigth GPlv3
* 
*/

#include <stdio.h>

#include <stdio.h>

int main() {
	char cadena[100];
	int n = 0;
	
	printf("Ingrese una cadena de texto: ");
	scanf("%s", cadena);
	
	while (cadena[n] != '\0') {
		n++;
	}
	
	printf("La longitud de la cadena es: %d", n);
	
	return 0;
}
	\end{lstlisting}
	
	\subsubsection{Ejecución}
	
	\begin{lstlisting}
	Ingrese una cadena de texto: asdasxfasdas
	La longitud de la cadena es: 12
	\end{lstlisting}
	
	\begin{lstlisting}
	Ingrese una cadena de texto: asfknjandcjascadbcndljnhvbaksnvhao
	La longitud de la cadena es: 34
	\end{lstlisting}
	
	\begin{lstlisting}
	Ingrese una cadena de texto: mfkdamflnsdmlfmfopdfkldskdopvsopvksdov
	La longitud de la cadena es: 38
	\end{lstlisting}
	
	\newpage
	
	\subsection{Problema 2}
	
	Programa que solicite al usuario una cadena de texto y muestre el total de palabras, total de letras, total de números y total de caracteres especiales.
	
	\subsubsection{Programa2.c}
	
	\begin{lstlisting}
		
/**
* @file Programa2.c
* @author Medina Martinez Jonathan Jason (jmedinam1702@alumno.ipn.mx)
* @brief 
* @version 0.1
* @date 2023-05-01
* 
* @copyrigth GPlv3
* 
*/

#include <stdio.h>

#define MAX 100

int main() {
	char cadena[MAX];
	int i, palabras = 0, letras = 0, numeros = 0, caracteres = 0;
	
	printf("Ingrese una cadena de texto: ");
	fgets(cadena, MAX, stdin);
	
	for (i = 0; cadena[i] != '\0'; i++) {
		if ((cadena[i] >= 'a' && cadena[i] <= 'z') || (cadena[i] >= 'A' && cadena[i] <= 'Z')) {
			letras++;
		} else if (cadena[i] >= '0' && cadena[i] <= '9') {
			numeros++;
		} else if (cadena[i] != ' ' && cadena[i] != '\n') {
			caracteres++;
		}
		
		if (cadena[i] == ' ') {
			palabras++;
		}
	}
	
	if (cadena[i - 1] != ' ') {
		palabras++;
	}
	
	printf("Total de palabras: %d\n", palabras);
	printf("Total de letras: %d\n", letras);
	printf("Total de numeros: %d\n", numeros);
	printf("Total de caracteres especiales: %d\n", caracteres);
	
	return 0;
}

	\end{lstlisting}
	
	\subsubsection{Ejecución}
	
	\begin{lstlisting}
	Ingrese una cadena de texto: juan lopez perez                     
	Total de palabras: 3
	Total de letras: 14
	Total de numeros: 0
	Total de caracteres especiales: 0
	\end{lstlisting}
	
	\begin{lstlisting}
	Ingrese una cadena de texto: mi cumple es el 24 de agosto :*D %3$
	Total de palabras: 9
	Total de letras: 21
	Total de numeros: 3
	Total de caracteres especiales: 4
	\end{lstlisting}	
	
	\newpage
	
	\subsection{Problema 3}
	
	Programa que solicite al usuario una cadena de texto y muestre la letra que se repite mas veces, incluyendo el numero de repeticiones.
	
	\subsubsection{Programa3.c}
	
	\begin{lstlisting}
/**
* @file Programa3.c
* @author Medina Martinez Jonathan Jason (jmedinam1702@alumno.ipn.mx)
* @brief 
* @version 0.1
* @date 2023-05-01
* 
* @copyrigth GPlv3
* 
*/

#include <stdio.h>

#define MAX 100

int main() {
	char cadena[MAX];
	int frecuencia[26] = {0};
	int repetida[26] = {0};
	int i, n = 0, m = -1;
	
	printf("Ingrese una cadena de texto: ");
	fgets(cadena, MAX, stdin);
	
	for (i = 0; cadena[i] != '\0'; i++) {
		if ((cadena[i] >= 'a' && cadena[i] <= 'z') || (cadena[i] >= 'A' && cadena[i] <= 'Z')) {
			if (cadena[i] >= 'a' && cadena[i] <= 'z') {
				frecuencia[cadena[i] - 'a']++;
			} else {
				repetida[cadena[i] - 'A']++;
			}
		}
	}
	
	for (i = 0; i < 26; i++) {
		if (frecuencia[i] + repetida[i] > n) {
			n = frecuencia[i] + repetida[i];
			m = i;
		}
	}
	
	if (m != -1) {
		printf("La letra que se repite mas veces es '%c', con %d repeticiones.\n", m + 'a', n);
	} else {
		printf("No se encontraron letras en la cadena ingresada.\n");
	}
	
	return 0;
}
	\end{lstlisting}
	
	\subsubsection{Ejecución}
	
	\begin{lstlisting}
	Ingrese una cadena de texto: julia lopez perez amanda alvarez
	La letra que se repite mas veces es 'a', con 6 repeticiones.
	\end{lstlisting}
	
	\begin{lstlisting}
	Ingrese una cadena de texto: 5549456 5459
	No se encontraron letras en la cadena ingresada.
	\end{lstlisting}
	
	\newpage
	
	\subsection{Problema 4}
	
	Programa que solicite al usuario una cadena de texto y muestre la palabra mas corta y la palabra mas larga.
	
	\subsubsection{Programa4.c}
	
	\begin{lstlisting}
/**
* @file Programa4.c
* @author Medina Martinez Jonathan Jason (jmedinam1702@alumno.ipn.mx)
* @brief 
* @version 0.1
* @date 2023-05-01
* 
* @copyrigth GPlv3
* 
*/

#include <stdio.h>

#define MAX 100

int main() {
	char str[MAX];
	char corta[MAX];
	char larga[MAX];
	int l = MAX;
	int m = 0;
	int n = 0;
	int i = 0;
	
	printf("Ingresa una cadena de texto: ");
	fgets(str, MAX, stdin);
	
	while (str[i] != '\0') {
		if (str[i] == ' ' || str[i] == '\n') {
			if (n < l) {
				l = n;
				sscanf(str + i - n, "%s", corta);
			}
			if (n > m) {
				m = n;
				sscanf(str + i - n, "%s", larga);
			}
			n = 0;
		} else {
			n++;
		}
		i++;
	}
	
	printf("La palabra mas corta es: %s\n", corta);
	printf("La palabra mas larga es: %s\n", larga);
	
	return 0;
}

	\end{lstlisting}
	
	\subsubsection{Ejecución}
	
	\begin{lstlisting}
	Ingresa una cadena de texto: juan perez lopez alfonso 
	La palabra mas corta es: juan
	La palabra mas larga es: alfonso
	\end{lstlisting}
	
	\newpage
	
	\subsection{Problema 5}
	
	Programa que solicite al usuario un carácter y muestre en pantalla si se trata de una letra mayúscula, minúscula, un numero o un carácter especial.
	
	\subsubsection{Programa5.c}
	
	\begin{lstlisting}
/**
* @file Programa5.c
* @author Medina Martinez Jonathan Jason (jmedinam1702@alumno.ipn.mx)
* @brief 
* @version 0.1
* @date 2023-05-01
* 
* @copyrigth GPlv3
* 
*/

#include <stdio.h>

int main() {
	char c;
	
	printf("Ingresa un caracter: ");
	scanf("%c", &c);
	
	if (c >= 'a' && c <= 'z') {
		printf("El caracter ingresado es una letra minuscula.\n");
	} else if (c >= 'A' && c <= 'Z') {
		printf("El caracter ingresado es una letra mayuscula.\n");
	} else if (c >= '0' && c <= '9') {
		printf("El caracter ingresado es un numero.\n");
	} else {
		printf("El caracter ingresado es un caracter especial.\n");
	}
	
	return 0;
}
	\end{lstlisting}
	
	\subsubsection{Ejecución}
	
	\begin{lstlisting}
	Ingresa un caracter: a
	El caracter ingresado es una letra minuscula.
	\end{lstlisting}
	
	\begin{lstlisting}
	Ingresa un caracter: 15
	El caracter ingresado es un numero.
	\end{lstlisting}
	
	\begin{lstlisting}
	Ingresa un caracter: A
	El caracter ingresado es una letra mayuscula.
	\end{lstlisting}
	
	\newpage
	
	\subsection{Problema 6}
	
	Programa que solicite al usuario una cadena de texto y muestre el numero de repeticiones de cada una de las letras. El programa no deberá distinguir entres mayúsculas y minúsculas.
	
	\subsubsection{Programa6.c}
	
	\begin{lstlisting}
/**
* @file Programa6.c
* @author Medina Martinez Jonathan Jason (jmedinam1702@alumno.ipn.mx)
* @brief 
* @version 0.1
* @date 2023-05-01
* 
* @copyrigth GPlv3
* 
*/

#include <stdio.h>

#define MAX 100

int main() {
	char str[MAX];
	int frecuencia[26] = {0};
	int i = 0;
	
	printf("Ingresa una cadena de texto: ");
	fgets(str, MAX, stdin);
	
	while (str[i] != '\0') {
		if (str[i] >= 'a' && str[i] <= 'z') {
			frecuencia[str[i] - 'a']++;
		} else if (str[i] >= 'A' && str[i] <= 'Z') {
			frecuencia[str[i] - 'A']++;
		}
		i++;
	}
	
	printf("Frecuencia de las letras:\n");
	for (int j = 0; j < 26; j++) {
		if (frecuencia[j] > 0) {
			printf("%c: %d\n", 'a' + j, frecuencia[j]);
		}
	}
	
	return 0;
}
	\end{lstlisting}
	
	\subsubsection{Ejecución}
	
	\begin{lstlisting}
	Ingresa una cadena de texto: sdasdasdsa
	Frecuencia de las letras:
	a   3
	d   3
	s   4
	\end{lstlisting}
	
	\newpage
	\section{Conclusión}
	
	En resumen, esta práctica sobre el manejo de cadenas en programación ha sido una excelente oportunidad para afianzar mis habilidades en el manejo de estructuras de datos. A través de la realización de diferentes ejercicios, he podido experimentar con diversas técnicas para manipular y procesar cadenas de texto.
	
\end{document}