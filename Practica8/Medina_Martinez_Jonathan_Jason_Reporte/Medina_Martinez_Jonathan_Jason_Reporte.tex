\documentclass{article}
\usepackage[utf8]{inputenc}
\usepackage[spanish]{babel}
\usepackage{graphicx}
\usepackage{geometry}
\usepackage{enumerate}
\usepackage{titlesec}
\usepackage{float}
\usepackage{listings}
\usepackage{xcolor}
\usepackage{amsmath}

\geometry{letterpaper, margin = 1.5cm}


\newcommand{\codefontsize}{\fontsize{12}{12}}

\lstset{
	language=C,
	basicstyle=\ttfamily\color{white},
	keywordstyle=\color{cyan},
	commentstyle=\color{lime},
	stringstyle=\color{red},
	numbers=left,
	numberstyle=\tiny\color{gray},
	numbersep=5pt,
	breaklines=true,
	showstringspaces=false,
	frame=tb,
	columns=fullflexible,
	backgroundcolor=\color{black},
}

%Datos de la Portada
\title{Introducción a la Programación \ Practica 8}
\author{Medina Martinez Jonathan Jason \ 2023640061}
\date{02 de junio del 2023}

\begin{document} %Inicio del Documento
	
	\fontsize{12}{16}\selectfont
	
	\begin{figure}[t] %Logos Portada
		
		\includegraphics[width=2.5 cm]{Logo1.jpeg}
		\hfill
		\includegraphics[width=3 cm]{Logo2.png}
		
	\end{figure}
	
	\maketitle %Titulo Portada
	\newpage
	
	\tableofcontents %Indice
	\newpage
	
	\section{Objetivo}
	
	Desarrollo de programas utilizando estructuras como aplicación para las bases de datos.
	
	\section{Introducción}
	
	El desarrollo de programas que utilicen estructuras como aplicación para las bases de datos es una habilidad fundamental en el ámbito de la programación. En esta práctica, nos centraremos en el desarrollo de un programa en lenguaje C que permita gestionar una lista de asistencia de alumnos. Para lograr esto, utilizaremos diferentes estructuras como Alumno, Profesor, Lista y Fecha. A través de un menú de opciones, los usuarios podrán registrar unidades de aprendizaje, grupos, fechas, profesores, así como agregar, editar y mostrar la lista de alumnos. Además, se utilizará un arreglo de tamaño dinámico para almacenar la lista de alumnos. A medida que avancemos en el desarrollo de esta práctica, exploraremos los conceptos clave de la programación estructurada y la gestión de bases de datos.
	
	\newpage
	
	\section{Desarrollo}
	
	\subsection{Lista de Alumnos}
	
	Desarrolle un programa en C que permita tener una lista de asistencia de alumnos.
	Para esto, deberá crear las siguientes estructuras:
	
	\begin{itemize}
		\item Alumno: Nombre, Numero de Boleta, Año de Ingreso, Carrera, Turno.
		\item Profesor: Nombre, Numero de Empleado, Turno.
		\item Lista: Fecha, Unidad de Aprendizaje, Grupo, Profesor, Lista de Alumnos.
		\item Fecha: Día, Mes, Año.
	\end{itemize}
	
	El programa deberá tener el siguiente menú:
	
	\begin{itemize}
		\item Registrar Unidad de Aprendizaje.
		\item Registrar Grupo.
		\item Registrar Fecha.
		\item Registrar Profesor.
		\item Agregar Alumno.
		\item Editar Alumno.
		\item Mostrar Lista.
		\item Salir.
	\end{itemize}
	
	La Lista de Alumnos deberá ser un arreglo de tamaño dinámico.
	
	\newpage
	
	\subsubsection{Lista.c}
	
	\begin{lstlisting}
/**
* @file Lista.c
* @author Medina Martinez Jonathan Jason (jmedinam1702@alumno.ipn.mx)
* @brief 
* @version 0.1
* @date 2023-06-03
* 
* @copyrigth GPlv3
* 
*/

#include <stdio.h>
#include <stdlib.h>
#include <string.h>

#define MAX_NOMBRE 50
#define MAX_CARRERA 50
#define MAX_ALUMNOS 100

typedef struct {
	char nombre[MAX_NOMBRE];
	int numeroBoleta;
	int anioIngreso;
	char carrera[MAX_CARRERA];
	char turno;
} Alumno;

typedef struct {
	char nombre[MAX_NOMBRE];
	int numeroEmpleado;
	char turno;
} Profesor;

typedef struct {
	int dia;
	int mes;
	int anio;
} Fecha;

typedef struct {
	Fecha fecha;
	char unidadAprendizaje[MAX_NOMBRE];
	char grupo;
	Profesor profesor;
	Alumno alumnos[MAX_ALUMNOS];
	int numAlumnos;
} Lista;

void registrarUnidadAprendizaje(Lista *lista);
void registrarGrupo(Lista *lista);
void registrarFecha(Lista *lista);
void registrarProfesor(Lista *lista);
void agregarAlumno(Lista *lista);
void editarAlumno(Lista *lista);
void mostrarLista(const Lista *lista);
void liberarMemoria(Lista *lista);

int main() {
	Lista lista;
	lista.numAlumnos = 0;
	
	int opcion;
	char opcionStr[10];
	
	do {
		printf("\n--- MENU ---\n");
		printf("1. Registrar Unidad de Aprendizaje\n");
		printf("2. Registrar Grupo\n");
		printf("3. Registrar Fecha\n");
		printf("4. Registrar Profesor\n");
		printf("5. Agregar Alumno\n");
		printf("6. Editar Alumno\n");
		printf("7. Mostrar Lista\n");
		printf("8. Salir\n");
		printf("Ingrese una opcion: ");
		fgets(opcionStr, sizeof(opcionStr), stdin);
		opcionStr[strcspn(opcionStr, "\n")] = '\0';
		opcion = atoi(opcionStr);
		
		switch (opcion) {
			case 1:
			registrarUnidadAprendizaje(&lista);
			break;
			case 2:
			registrarGrupo(&lista);
			break;
			case 3:
			registrarFecha(&lista);
			break;
			case 4:
			registrarProfesor(&lista);
			break;
			case 5:
			agregarAlumno(&lista);
			break;
			case 6:
			editarAlumno(&lista);
			break;
			case 7:
			mostrarLista(&lista);
			break;
			case 8:
			liberarMemoria(&lista);
			printf("Hasta luego!\n");
			break;
			default:
			printf("Opcion no valida. Intente de nuevo.\n");
			break;
		}
	} while (opcion != 8);
	
	return 0;
}

void registrarUnidadAprendizaje(Lista *lista) {
	printf("\n--- Registrar Unidad de Aprendizaje ---\n");
	printf("Ingrese el nombre de la unidad de aprendizaje: ");
	fgets(lista->unidadAprendizaje, sizeof(lista->unidadAprendizaje), stdin);
	lista->unidadAprendizaje[strcspn(lista->unidadAprendizaje, "\n")] = '\0';
}

void registrarGrupo(Lista *lista) {
	printf("\n--- Registrar Grupo ---\n");
	printf("Ingrese el grupo: ");
	scanf(" %c", &(lista->grupo));
	getchar();
}

void registrarFecha(Lista *lista) {
	printf("\n--- Registrar Fecha ---\n");
	printf("Ingrese el dia: ");
	scanf("%d", &(lista->fecha.dia));
	printf("Ingrese el mes: ");
	scanf("%d", &(lista->fecha.mes));
	printf("Ingrese el anio: ");
	scanf("%d", &(lista->fecha.anio));
}

void registrarProfesor(Lista *lista) {
	printf("\n--- Registrar Profesor ---\n");
	printf("Ingrese el nombre del profesor: ");
	fgets(lista->profesor.nombre, sizeof(lista->profesor.nombre), stdin);
	lista->profesor.nombre[strcspn(lista->profesor.nombre, "\n")] = '\0';
	printf("Ingrese el numero de empleado: ");
	scanf("%d", &(lista->profesor.numeroEmpleado));
	printf("Ingrese el turno del profesor: ");
	scanf(" %c", &(lista->profesor.turno));
}

void agregarAlumno(Lista *lista) {
	printf("\n--- Agregar Alumno ---\n");
	if (lista->numAlumnos < MAX_ALUMNOS) {
		Alumno *nuevoAlumno = &(lista->alumnos[lista->numAlumnos]);
		
		printf("Ingrese el nombre del alumno: ");
		fgets(nuevoAlumno->nombre, sizeof(nuevoAlumno->nombre), stdin);
		nuevoAlumno->nombre[strcspn(nuevoAlumno->nombre, "\n")] = '\0';
		printf("Ingrese el numero de boleta: ");
		scanf("%d", &(nuevoAlumno->numeroBoleta));
		printf("Ingrese el anio de ingreso: ");
		scanf("%d", &(nuevoAlumno->anioIngreso));
		printf("Ingrese la carrera del alumno: ");
		fgets(nuevoAlumno->carrera, sizeof(nuevoAlumno->carrera), stdin);
		nuevoAlumno->carrera[strcspn(nuevoAlumno->carrera, "\n")] = '\0';
		printf("Ingrese el turno del alumno: ");
		scanf(" %c", &(nuevoAlumno->turno));
		
		lista->numAlumnos++;
	} else {
		printf("No se pueden agregar mas alumnos. Limite alcanzado.\n");
	}
}

void editarAlumno(Lista *lista) {
	printf("\n--- Editar Alumno ---\n");
	int indice;
	printf("Ingrese el indice del alumno a editar (1-%d): ", lista->numAlumnos);
	scanf("%d", &indice);
	
	if (indice >= 1 && indice <= lista->numAlumnos) {
		Alumno *alumno = &(lista->alumnos[indice - 1]);
		
		printf("Ingrese el nombre del alumno: ");
		fgets(alumno->nombre, sizeof(alumno->nombre), stdin);
		alumno->nombre[strcspn(alumno->nombre, "\n")] = '\0';
		printf("Ingrese el numero de boleta: ");
		scanf("%d", &(alumno->numeroBoleta));
		printf("Ingrese el anio de ingreso: ");
		scanf("%d", &(alumno->anioIngreso));
		printf("Ingrese la carrera del alumno: ");
		fgets(alumno->carrera, sizeof(alumno->carrera), stdin);
		alumno->carrera[strcspn(alumno->carrera, "\n")] = '\0';
		printf("Ingrese el turno del alumno: ");
		scanf(" %c", &(alumno->turno));
		
		printf("Alumno editado con exito.\n");
	} else {
		printf("Indice de alumno invalido.\n");
	}
}

void mostrarLista(const Lista *lista) {
	printf("\n--- Mostrar Lista ---\n");
	printf("Unidad de Aprendizaje: %s\n", lista->unidadAprendizaje);
	printf("Grupo: %c\n", lista->grupo);
	printf("Fecha: %02d/%02d/%d\n", lista->fecha.dia, lista->fecha.mes, lista->fecha.anio);
	printf("Profesor: %s\n", lista->profesor.nombre);
	printf("Numero de Empleado: %d\n", lista->profesor.numeroEmpleado);
	printf("Turno del Profesor: %c\n", lista->profesor.turno);
	
	printf("\nAlumnos:\n");
	for (int i = 0; i < lista->numAlumnos; i++) {
		const Alumno *alumno = &(lista->alumnos[i]);
		printf("%d. %s, Boleta: %d, Anio de Ingreso: %d, Carrera: %s, Turno: %c\n",
		i + 1, alumno->nombre, alumno->numeroBoleta, alumno->anioIngreso, alumno->carrera, alumno->turno);
	}
}

void liberarMemoria(Lista *lista) {
	if (lista->alumnos != NULL) {
		lista->numAlumnos = 0;
		memset(lista->alumnos, 0, sizeof(lista->alumnos));
	}
}
	\end{lstlisting}
	
	\newpage
	
	\section{Conclusión}
	
	En conclusión, esta práctica nos ha permitido adquirir conocimientos y habilidades fundamentales en el desarrollo de programas que utilizan estructuras como aplicación para las bases de datos. A lo largo del proceso, hemos aprendido a diseñar y utilizar estructuras como Alumno, Profesor, Lista y Fecha para gestionar una lista de asistencia de alumnos. Además, hemos implementado un menú de opciones que ofrece funcionalidades como el registro de unidades de aprendizaje, grupos, fechas y profesores, así como la capacidad de agregar, editar y mostrar la lista de alumnos. También hemos utilizado un arreglo de tamaño dinámico para adaptarnos a las necesidades cambiantes de la lista de alumnos. Esta práctica ha sido un paso importante en nuestro aprendizaje de la programación estructurada y la gestión de bases de datos, y nos ha preparado para enfrentar desafíos más complejos en el futuro.
	
\end{document}